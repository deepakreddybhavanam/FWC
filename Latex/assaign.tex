\documentclass[10pt,-letter paper]{article}
\usepackage[left=1in, right=0.75in, top=1in, bottom=0.75in]{geometry}
\usepackage{graphicx} % Required for inserting images
\usepackage{siunitx}
\usepackage{setspace}
\usepackage{gensymb}
\usepackage{xcolor}
\usepackage{caption}
%\usepackage{subcaption}
\doublespacing
\singlespacing
\usepackage[none]{hyphenat}
\usepackage{amssymb}
\usepackage{relsize}
\usepackage[cmex10]{amsmath}
\usepackage{mathtools}
\usepackage{amsmath}
\usepackage{commath}
\usepackage{amsthm}
\interdisplaylinepenalty=2500
%\savesymbol{iint}
\usepackage{txfonts}
%\restoresymbol{TXF}{iint}
\usepackage{wasysym}
\usepackage{amsthm}
\usepackage{mathrsfs}
\usepackage{txfonts}
\let\vec\mathbf{}
\usepackage{stfloats}
\usepackage{float}
\usepackage{cite}
\usepackage{cases}
\usepackage{subfig}
%\usepackage{xtab}
\usepackage{longtable}
\usepackage{multirow}
%\usepackage{algorithm}
\usepackage{amssymb}
%\usepackage{algpseudocode}
\usepackage{enumitem}
\usepackage{mathtools}
%\usepackage{eenrc}
%\usepackage[framemethod=tikz]{mdframed}
\usepackage{listings}
%\usepackage{listings}
\usepackage[latin1]{inputenc}
%%\usepackage{color}{   
%%\usepackage{lscape}
\usepackage{textcomp}
\usepackage{titling}
\usepackage{hyperref}
%\usepackage{fulbigskip}   
\usepackage{tikz}
\usepackage{graphicx}
\lstset{
  frame=single,
  breaklines=true
}
\let\vec\mathbf{}
\usepackage{enumitem}
\usepackage{graphicx}
\usepackage{siunitx}
\let\vec\mathbf{}
\usepackage{enumitem}
\usepackage{graphicx}
\usepackage{enumitem}
\usepackage{tfrupee}
\usepackage{amsmath}
\usepackage{amssymb}
\usepackage{mwe} % for blindtext and example-image-a in example
\usepackage{wrapfig}
\graphicspath{{figs/}}
\providecommand{\cbrak}[1]{\ensuremath{\left\{#1\right\}}}
\providecommand{\brak}[1]{\ensuremath{\left(#1\right)}}
\newcommand{\sgn}{\mathop{\mathrm{sgn}}}
\providecommand{\abs}[1]{\left\vert#1\right\vert}
\providecommand{\res}[1]{\Res\displaylimits_{#1}} 
\providecommand{\norm}[1]{\left\lVert#1\right\rVert}
%\providecommand{\norm}[1]{\lVert#1\rVert}
\providecommand{\mtx}[1]{\mathbf{#1}}
\providecommand{\mean}[1]{E\left[ #1 \right]}
\providecommand{\fourier}{\overset{\mathcal{F}}{ \rightleftharpoons}}
%\providecommand{\hilbert}{\overset{\mathcal{H}}{ \rightleftharpoons}}
\providecommand{\system}{\overset{\mathcal{H}}{ \longleftrightarrow}}
 %\newcommand{\solution}[2]{\textbf{Solution:}{#1}}
%\newcommand{\solution}{\noindent \textbf{Solution: }}
\newcommand{\cosec}{\,\text{cosec}\,}
\providecommand{\dec}[2]{\ensuremath{\overset{#1}{\underset{#2}{\gtrless}}}}
\newcommand{\myvec}[1]{\ensuremath{\begin{pmatrix}#1\end{pmatrix}}}
\newcommand{\myaugvec}[2]{\ensuremath{\begin{amatrix}{#1}#2\end{amatrix}}}
\newcommand{\mydet}[1]{\ensuremath{\begin{vmatrix}#1\end{vmatrix}}}
\title{MATHEMATICS}
\author{SECTION A}
\date{\today}
\begin{document}
\maketitle
\begin{enumerate}
\section{Vectors}
\item Show that the function $f:\mathbb{R}\rightarrow \mathbb{R}$ defined by $f\brak{x} = \frac{x}{x^{2}+1}, \forall \mathbf{x}\in \mathbb{R}$ is neither one-one nor onto. Also,if $g:\mathbb{R} \rightarrow \mathbb{R}$ is defined as $g\brak{x}=2x-1$,find $fog\brak{x}$.
\item  Find the distance of the point \brak{-1,-5,-10} from the point of intersection of the line $\overrightarrow{\mathbf{r}}=2\hat{i}-\hat{j}+2\hat{k} + \lambda\brak{3\hat{i}+4\hat{j}+2\hat{k}}$ and the plane $\overrightarrow{\mathbf{r}}\cdot\brak{\hat{i}-\hat{j}+\hat{k}}=5$.
\section{Matrices}
\item Using properties of determinants,prove that 
	\begin{align*}
		\mydet{1 & 1 & 1+3x \\ 1+3y & 1 & 1 \\ 1 & 1+3z & 1 }=9\brak{3xyz+xy+yz+zx}
	\end{align*}
\item If $A=\myvec{2 & -3 & 5 \\ 3 & 2 & -4 \\ 1 & 1 & -2}$, Find the $A^{-1}$. Use it ton solve the system of equations 
	\begin{align*}
		2x-3y+5z=11 \\
		3x-2y-4z=-5\\
		x+y-2z=-3
	\end{align*}
\item Using elementary row transformations, find the inverse of the matrix $A=\myvec{1 & 2 & 3 \\ 2 & 5 & 7 \\ -2 & -4 & -5}$.


   \section{Differentiation}
 \item If $y=\sin\brak{\sin x}$, prove that 
		\begin{align*}
			\frac{d^{2}y}{dx^{2}}+\tan x \frac{dy}{dx}+y\cos^{2}x=0
		\end{align*}
	
 \item If $x=a\brak{20-\sin 20}$ and $y=a\brak{1-\cos 20}$, find $\frac{dy}{dx}$ when $\theta=\frac{\pi}{3}$
\section{Integration}
\item Evaluate: 
\begin{align*}
	\int_{0}^{\frac{\pi}{4}}\frac{\sin x+\cos x}{16+9 \sin 2x}.1
\end{align*}
\item Evaluate 
	\begin{align*}
		\int_{1}^{3} \brak{x^{2}+3x+e^{x}}dx,
	\end{align*}
	as the limit of the sum.
\item Find:
	\begin{align*}
		\int\frac{2\cos x}{\brak{1-\sin x}\brak{1+\sin^{2}x}}dx
	\end{align*}
 
 \section{Probability}
\item  Suppose a girl throws a die. If she gets $1$ or $2$, she tosses a coin three times and notes the number of tails. If she gets $3$, $4$, $5$ or $6$, she tosses a coin once and notes whether a 'head' or 'tail' is obtained. If she obtained exactly one 'tail', what is the probability that she threw $3$, $4$, $5$ or $6$ with the die ? 



\end{enumerate}
\end{document}
